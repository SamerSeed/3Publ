\documentclass[main.tex]{subfiles}

\begin{document}
\section{Кусочно-параллельные параметризации}
\subsection{Доказательства замкнутости $NS$ и $NS_0$}.
\begin{lemma}
    Класс $NS$ замкнут и предполон.
\end{lemma}
\begin{proof}
    Докажем замкнутость. \\
    Пусть $f(x_1, x_2, \ldots, x_n), g(y_1. y_2, \ldots, y_k) \in NS$. 
    Пусть $h(x_1, x_2, \ldots, x_{n-1}) = f(x_1, x_2, \ldots, x_{n-1}, x_1)$. Если для любой несюръективной парамметризации $t_j(t)$,
    $f(t_1(t), t_2(t), \ldots, t_n(t))$ не сюръективна, то значит она не сюрективна для несюръективных, парамметризаций в которых $t_1 = t_n$,
    а значит $f(t_1(t), t_2(t), \ldots, t_1(t)) = h(t_1(t), t_2(t), \ldots, t_{n-1}(t))$ не сюрективна.\\
    Пусть $u(x_1, x_2, \ldots, x_{n-1}, y_1, y_2, \ldots, y_k) = f(x_1, x_2, \ldots, x_{n-1}, g(y_1, y_2, \ldots, y_k))$. Рассмотрим одноместную несюръективную парамметризацию.
\end{proof}

\subsection{Функции с малой несюръективной частью}
Тут от том, что у нас всегда есть $x + \theta(x)$ и $x-\theta(x) + \epsilon \theta(x+C)$

\subsection{О перестановке областей значений параметризацией}
Тут надо показать, что функциями с малой несюръективной частью можно переставить области знвчений в удобном нам порядке.
  
\end{document}