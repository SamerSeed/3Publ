\documentclass[main.tex]{subfiles}

\begin{document}
\section{Кусочно-параллельные параметризации}
\subsection{Доказательства замкнутости $NS$ и $NS_0$}.
Тут просто

\subsection{Функции с малой несюръективной частью}
Тут от том, что у нас всегда есть $x + \theta(x)$ и $x-\theta(x) + \epsilon \theta(x+C)$

\subsection{О перестановке областей значений параметризацией}
Тут надо показать, что функциями с малой несюръективной частью можно переставить области знвчений в удобном нам порядке.
  
\end{document}