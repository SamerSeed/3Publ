\documentclass[a4paper]{article}

\usepackage[utf8]{inputenc}
\usepackage[english, russian]{babel}
\usepackage{amsthm}
\usepackage{amsmath}
\usepackage{amssymb}
\usepackage{array}
\usepackage{subfiles}

\theoremstyle{plain}
\newtheorem{theorem}{Теорема}
\newtheorem{lemma}{Лемма}
\newtheorem{corollary}{Следствие}
\newtheorem{proposition}{Предложение}
\newtheorem{assertion}{Утверждение}

\theoremstyle{definition}
\newtheorem{definition}{Определение}
\newtheorem{remark}{Замечание}
\newtheorem{example}{Пример}
\sloppy

\begin{document}
\title{О выразимости кусочно-постоянных функций в пространстве кусочно-параллельных}

\subfile{intro.tex}
\subfile{defs.tex}
\subfile{sectNs.tex}



\begin{thebibliography}{99}

  \bibitem{Polov1}
  В.~С.~Половников
   О задаче проверки функциональной полноты в классе кусочно-параллельных функций
   Вестн. Моск. ун-та. Сер.~1. Матем., мех.
   2008
   6
   31--35
  {http://mi.mathnet.ru/vmumm990}
  {http://www.ams.org/mathscinet-getitem?mr=2517018}
  {https://zbmath.org/?q=an:1304.94135}

  \bibitem{Polov4}
  Половников В.С.
  О нелинейной сложности нейронных схем Мак-Каллока-Питтса.
   М., Интеллектуальные системы.
   11
   2007
   261-275

  \bibitem{Otro1}
   Отрощенко А.Д.
   Классы кусочно-параллельных функций, содержащие все одноместные.
    М., Интеллектуальные системы.
   4
   2020
   57-74

\end{thebibliography}


\end{document}
% Конец статьи
