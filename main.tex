\documentclass[a4paper]{article}

\usepackage[utf8]{inputenc}
\usepackage[english, russian]{babel}
\usepackage{amsthm}
\usepackage{amsmath}
\usepackage{amssymb}
\usepackage{array}

\theoremstyle{plain}
\newtheorem{theorem}{Теорема}
\newtheorem{lemma}{Лемма}
\newtheorem{corollary}{Следствие}
\newtheorem{proposition}{Предложение}
\newtheorem{assertion}{Утверждение}

\theoremstyle{definition}
\newtheorem{definition}{Определение}
\newtheorem{remark}{Замечание}
\newtheorem{example}{Пример}

\sloppy

\begin{document}
\title{О выразимости кусочно-постоянных функций в пространстве кусочно-параллельных}

\section{Определение кусочно-параллельной \\ функции}
В соответствии с \cite{Polov4}, мы рассматриваем класс
$PP$
кусочно-параллельных функций, которые строятся из линейных функций \\
$a_1 x_1+...+a_n x_n+a_0: R^n \to R, a_i\in R, i=0,1,...,n, n\in \mathbb{N}$
и функции Хэвисайда \\
$\theta(x) = \begin{cases} 1, x \geq 0 \\ 0, x< 0 \end{cases}$
с использованием операций суперпозиции.
Как показано в \cite{Polov4}, функция
$f$ из $PP$ может быть представлена в следующем виде:
$f = f_L + f_{PC}$,
где $f_L$ -линейная функция, а
$f_{PC}$ - кусочно-постоянная функция.
Будем обозначать
$\vec{a} = (a_1, a_2, \ldots, a_n)$, \\
$(a, \vec{b}) = (a, b_1, b_2, \ldots, b_n)$,\\
$\langle \vec{a}, \vec{b} \rangle =  \sum_{i = 1}^{n} a_i b_i$,\\
$\vec{e}_i = (0, \ldots, 0, 1, \ldots, 0)$, $\vec{e}_0 = (1, 1, 1, \ldots, 1, 1)$\\
$\partial A$ граница множества $A$\\
В соответствии с \cite{Polov1}, кусочно-параллельная функция имеет вид
\begin{equation}
  f(\vec{x}) = \langle \vec{a}_0, \vec{x} \rangle + \sum_{i = 1}^{s} d_i \theta(\sum_{j = 1}^{k} \chi(sgn(\langle \vec{a}_j, \vec{x} \rangle + c_j) = \sigma_{ij}) - k).
\end{equation}
где
$\sigma_{ij} \in \{-1, 0, 1\}$,
$\chi(A)=\begin{cases} 1, \mbox{условие } A \mbox{ выполнено} \\ 0, \mbox{условие } A \mbox{ не выполнено}  \end{cases}$.\\
В дальнейшем, вместо
$\sum_{i = 1}^{s} d_i \theta(\sum_{j = 1}^{k} \chi(sgn(\langle \vec{a}_j, \vec{x} \rangle + c_j) = \sigma_{ij}) - k)$
мы будем писать
$NL_f(\vec{x})$.
Пусть $l = max |d_i|$.
Иногда, мы будем писать
$NL_{f}^{l}(\vec{x})$.
Нижний индекс мы также иногда будем опускать, обозначая таким образом произвольную кусочно-постоянную функцию,
максимум модуля которой не больше $l$.
Также будем теперь обозначать
$f_{\epsilon}( \vec{x}) =\epsilon f( \frac{\vec{x}}{\epsilon}).$
\\
Будем называть множество значений аргумента соответствующую определенному
$d_i$ в дальнейшем носителем сигнатуры i, а сам $d_i$ - сдвигом.
Носитель сигнатуры, неограниченный хотя бы с одной стороны по каждой из координат, будем называть нограниченным.
Плоскости разделяющие носители сигнатуры будем называть разрезами.
Мы будем рассматривать дальше кусочно-параллельные функции с конечным числом сдвигов.\\
Обозначим множество функций с линейной частью, зависящей от не более чем одной переменной $NLL_1$.
В соответствии с \cite{Otro1}, это замкнутый предполный класс функций.
С-финитно-линейные функции в соответствии с В. С. Половниковым \cite{Polov1},
будем обозначать как $FL$. Напомним:
\begin{definition}
  Кусочно-параллельнаая функция $f(x_1, x_2, \ldots, x_n)$, такая, что $\forall a_i, b_i \exists C, A, B$, такие, что при
  $|t| > C: f(a_1 t + b_1, a_2 t + b_2, \ldots, a_n t + b_n) = A t + B$ называется С-финитно-линейной.
\end{definition}
Класс С-финитно-линейных функций замкнут и предполон.
Само утверждение, которое будет доказано, звучит следующим образом:
\begin{theorem} \label{th:main}
  Пусть $U = L_1 \cup U_{add}$, где $L_1$ - все линейные одноместные функции, а $U_{add}$ - конечное множество кусочно-параллельных.
  Замыкание $U$ содержит все кусочно-постоянные функции тогда, и только тогда, когда $U \not\subseteq NLL_1, FL$
\end{theorem}

\section{Кусочно-параллельные параметризации}
\subsection{Определения}
  \begin{definition}
    Пусть $t_i(x)$ - набор одноместных кусочно-параллельных Функций. 
    Тогда для $f(x_1, x_2, \ldots, x_n)$ назовем $f^{p}_{t_1, t_2, \ldots, t_n}(x) = f(t_1(x), t_2(x), \ldots, t_n(x))$
    одноместной параметризацией $f$ набором $t_i(x)$.
  \end{definition}
  \begin{definition}
    Функцию $f(R^n)$ назовем 0-сюръективной, если $\mu(R / f(R^n)) = 0$.
  \end{definition}
      В мы можем считать, что мы множество в определении выше всегда измеримо, т.к. имеем дело с кусочно-параллельными функциями.
  \begin{definition}
  Множество функций, таких, что дюбая одноместной парамметризации состоящией из не сюръективных функций не сюръективна назовем несильно сюръективными функциями.
  \end{definition}
  \begin{definition}
  Множество функций, таких, что дюбая одноместной парамметризации состоящией из не 0-сюръективных функций не 0-сюръективна назовем несильно 0-сюръективными функциями.
  \end{definition}



\begin{thebibliography}{99}

  \bibitem{Polov1}
  В.~С.~Половников
   О задаче проверки функциональной полноты в классе кусочно-параллельных функций
   Вестн. Моск. ун-та. Сер.~1. Матем., мех.
   2008
   6
   31--35
  {http://mi.mathnet.ru/vmumm990}
  {http://www.ams.org/mathscinet-getitem?mr=2517018}
  {https://zbmath.org/?q=an:1304.94135}

  \bibitem{Polov4}
  Половников В.С.
  О нелинейной сложности нейронных схем Мак-Каллока-Питтса.
   М., Интеллектуальные системы.
   11
   2007
   261-275

  \bibitem{Otro1}
   Отрощенко А.Д.
   Классы кусочно-параллельных функций, содержащие все одноместные.
    М., Интеллектуальные системы.
   4
   2020
   57-74

\end{thebibliography}


\end{document}
% Конец статьи
